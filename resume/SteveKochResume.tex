%______________________________________________________________________________________________________________________
% @brief    LaTeX2e Resume for Steven J. Koch (based on template from http://linux.dsplabs.com.au/resume-writing-example-latex-template-linux-curriculum-vitae-professional-cv-layout-format-text-p54/)
\documentclass[margin,line]{resume}


%______________________________________________________________________________________________________________________
\begin{document}
\name{\Large Steven J. Koch}
\begin{resume}

    %__________________________________________________________________________________________________________________
    % Contact Information
    \section{\mysidestyle Contact\\Information}

    Research Data Scientist, University Libraries                \hfill mobile: 505-263-7400          \vspace{0mm}\\\vspace{0mm}%
    Visiting Applications Scientist, CARC                                \hfill email: stevekochscience@gmail.com          \vspace{0mm}\\\vspace{0mm}%
    University of New Mexico                                         \hfill LinkedIn: www.linkedin.com/in/stevekoch  \vspace{0mm}\\\vspace{0mm}%
    Albuquerque, NM                                         \hfill CAREERS 2.0: careers.stackoverflow.com/sjkode  \vspace{0mm}\\\vspace{-4.5mm}%


    %__________________________________________________________________________________________________________________
    % Objective
%    \section{\mysidestyle Objective}
%
%    To join a team or provide consulting that leverages my talents in parallel computing, coding, data acquisition, automation, algorithm development, data analysis, and visualization.
%
    %__________________________________________________________________________________________________________________
    % Programming Parallel Computing Skills
    \section{\mysidestyle Coding} 

    TORQUE, Gnu Parallel, LabVIEW / NI-DAQmx (extensive experience), Python, R, C, Java, LaTeX, git, bash \newline stackoverflow reputation 382

    %__________________________________________________________________________________________________________________
    % Other Skills
    \section{\mysidestyle Other skills} 

    Image processing / tracking, hardware automation, molecular biology, microscopy, metal machining, Monte Carlo and Markov chain analysis, teaching and mentoring  
 
 
    %__________________________________________________________________________________________________________________
    % Education
    \section{\mysidestyle Education}

    \textbf{Cornell University}, Ithaca, NY \vspace{2mm}\\\vspace{1mm}%
    \textsl{Ph.D., Physics (Biophysics minor)} \hfill \textbf{ May 2003}\vspace{-3mm}\\\vspace{-1mm}%
    \begin{list2}
        \item Dissertation: Probing protein-DNA interactions by unzipping \\
        single DNA molecules with a laser trapping microscope
        \item Advisor:  Professor Michelle D. Wang
    \end{list2}\vspace{-1.5mm}

    \textsl{M.S., Physics} \hfill \textbf{2000}\\\vspace{0mm}%

    \textbf{University of Michigan}, Ann Arbor, MI \vspace{2mm}\\\vspace{1mm}%
    \textsl{B.S., Honors Physics} \hfill \textbf{1996}\vspace{-3mm}\\\vspace{-1mm}%
    
    %__________________________________________________________________________________________________________________
    % Professional Experience
    \section{\mysidestyle Recent\\Experience}

    \textbf{University of New Mexico}, Albuquerque, NM \vspace{2mm}\\\vspace{1mm}%
    \textsl{Research Data Scientist (University Libraries)} \hfill \textbf{June 2013 -- Present}\\
    \textsl{Visiting Applications Scientist (CARC)} \hfill \textbf{January 2014 -- Present}\\
    Leading a pilot project within the University Libraries to help connect campus researchers with high performance computing (HPC). I work one-on-one with researchers, helping them learn the basics of the HPC environment, the nuances of the various machines, and adapting their code to run in parallel on CARC machines. Specific emphasis is on non-traditional supercomputing problems that do not need to have intrinsically parallel code. This is part of broader role as team member of growing Research and Data Services at the UNM Libraries.

    \textbf{University of New Mexico}, Albuquerque, NM \vspace{2mm}\\\vspace{1mm}%
    \textsl{Assistant Professor} \hfill \textbf{August 2006 -- May 2013}\\
    One large grant (DTRA, \$1.5M, co-PI with Atlas), state of the art optical tweezers and automated kinesin gliding motility assays. Four Ph.D. students graduated. Mentored 8 undergraduate researchers. Taught more than 700 students, mostly undergraduate courses, with excellent reviews.  Open-science advocate.

    \textbf{Sandia National Labs}, Albuquerque, NM \vspace{2mm}\\\vspace{1mm}%
    \textsl{CINT Distinguished Postdoctoral Fellow, Appointee} \hfill \textbf{2003 -- 2006}\\
    Implemented wide array of collaborative biophysics projects across Sandia and LANL / CINT.  Major publications in MEMS (Applied Physics Letters) and Kinesin (Fungal Genetics and Biology)

    
    %__________________________________________________________________________________________________________________
    % Publications
    \section{\mysidestyle Scientific Publications}
    
    Available on my Google Scholar page (Steven J. Koch http://goo.gl/kszZ3).  Highly-cited publications in Biophysical Journal (2002, 90 citations), Physical Review Letters (2003, 47 citations), Advanced Materials (2008, 26 citations), Nano Letters (2008, 18 citations), Applied Physics Letters (2006, 17 citations), Fungal Genetics and Biology (2007, 7 citations).
    
    
    %__________________________________________________________________________________________________________________
    % Honors and Awards
    \section{\mysidestyle Honors and\\Awards} 

    Addgene Resource Sharing Award, CINT postdoctoral fellowship, US Dept. Ed. GAANN TA/RA fellow, Honorable Mention NSF Graduate Research Fellowship, U. Michigan: Sigma Pi Sigma, Phi Beta Kappa, James B. Angell Scholar, and Sharon Naughton-Briggs Memorial Scholarship.     


%______________________________________________________________________________________________________________________
\end{resume}
\end{document}


%______________________________________________________________________________________________________________________
% EOF

