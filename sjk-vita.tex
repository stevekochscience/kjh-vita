%%% A template to produce a nice-looking Curriculum Vitae.
%%% Kieran Healy <kjhealy@gmail.com>
%%% Most recent version is at http://kjhealy.github.com/kjh-vita
%%% Steve Koch <stevekochscience@gmail.com> followed Kieran's template
%%% Most recent version at http://stevekochscience.github.com/kjh-vita/
%%% ------------------------------------------------------------------------
%%% Requirements (should be included in a modern tex distribution):
%%% ------------------------------------------------------------------------
%%% xelatex
%%% fontspec.sty
%%% hyperrref.sty
%%% xunicode.sty
%%% color.sty
%%% url.sty
%%% fancyhdr.sty
%%%
%%% ------------------------------------------------------------------------
%%% Optional
%%% ------------------------------------------------------------------------
%%% git
%%% vc.sty
%%% revnum.sty
%%% Fonts
%%%
%%% ------------------------------------------------------------------------
%%% Note
%%%------------------------------------------------------------------------
%%% Because this is a hand-tweaked file, be on the look out for \medksip, 
%%% \bigskip and \newpage commands here and there, which are used to balance
%%% the layout or avoid widows & orphans, etc. You should of course add or 
%%% remove these as needed.
%%%------------------------------------------------------------------------

%%% dash symbols for SJK use - vs. – vs. —

\documentclass[11pt]{article}

%%%------------------------------------------------------------------------
%%% Metadata
%%%------------------------------------------------------------------------

%% Change as needed. Or just add me as a coauthor. Only some of these are 
%% used below in the hyperref declaration and address banner section.
\def\myauthor{Steven J. Koch}
\def\mycoauthor{Kieran Healy}
\def\mytitle{Vita}
\def\mycopyright{\myauthor}
\def\mykeywords{}
\def\mybibliostyle{plain}
\def\mybibliocommand{}
\def\mysubtitle{}
\def\myaffiliation{University of New Mexico}
\def\myaddress{University Libraries and Center for Advanced Research Computing}
\def\myemail{stevekochscience@gmail.com}
\def\myweb{http://www.linkedin.com/in/stevekoch/}
\def\mycell{(505) 263-7400}
\def\mygoogle{(505) 750-3279}
\def\myversion{}
\def\myrevision{}


\def\myaffiliation{University of New Mexico}
\def\myauthor{Steven J. Koch}
\date{} % not used (revision control instead)
\def\mykeywords{Steven, Koch, Steven Koch, Steven J. Koch, Vita, CV, Resume, Data Science, Data Scientist, Physics, Microscopy, Molecular Biology, Parallel Computing}

%%%------------------------------------------------------------------------  
%%% Git version tracking 
%%%------------------------------------------------------------------------

%% If you don't use git or the vc package (from CTAN), comment this out.
%% If you comment it out, be sure to remove the \rfoot comment below, too.
%% \immediate\write18{sh ./vc} (can't figure out how to generate the vc.tex file
%%% This file has been generated by the vc bundle for TeX.
%%% Do not edit this file!
%%%
%%% Define Git specific macros.
\gdef\GITHash{9ff13d9b81e01641d3e05b495802089e1a2cd71f}%
\gdef\GITAbrHash{9ff13d9}%
\gdef\GITParentHashes{d0d54f9a5e5aff8fcc1df4e3d875fa3d80534a3a}%
\gdef\GITAbrParentHashes{d0d54f9}%
\gdef\GITAuthorName{Steve Koch}%
\gdef\GITAuthorEmail{stevekochscience@gmail.com}%
\gdef\GITAuthorDate{2013-02-17 21:13:01 -0700}%
\gdef\GITCommitterName{Steve Koch}%
\gdef\GITCommitterEmail{stevekochscience@gmail.com}%
\gdef\GITCommitterDate{2013-02-17 21:13:01 -0700}%
%%% Define generic version control macros.
\gdef\VCRevision{\GITAbrHash}%
\gdef\VCAuthor{\GITAuthorName}%
\gdef\VCDateRAW{2013-02-17}%
\gdef\VCDateISO{2013-02-17}%
\gdef\VCDateTEX{2013/02/17}%
\gdef\VCTime{21:13:01 -0700}%
\gdef\VCModifiedText{\textcolor{red}{with local modifications!}}%
%%% Assume clean working copy.
\gdef\VCModified{0}%
\gdef\VCRevisionMod{\VCRevision}%


%%%------------------------------------------------------------------------
%%% Required style files
%%%------------------------------------------------------------------------
\usepackage{url,fancyhdr}
%%\usepackage{revnum} % for reverse-numbered publications (revnumerate environment) if needed.

%% needed for xelatex to work
\usepackage{fontspec}
\usepackage{xunicode}

%% color for the links 
\usepackage[usenames,dvipsnames]{xcolor}

%% hyperlinks
\usepackage[xetex, 
	colorlinks=true,
	urlcolor=BlueViolet,
	plainpages=false,
  	pdfpagelabels,
  	bookmarksnumbered,
  	pdftitle={\mytitle},
  	pagebackref,
  	pdfauthor={\myauthor},
  	pdfkeywords={\mykeywords}
  	]{hyperref}

%%%------------------------------------------------------------------------
%%% Document
%%%------------------------------------------------------------------------
\begin{document}

%% Choose fonts for use with xelatex
%% Minion and Myriad are widely available, from Adobe. 
%% Pragmata is available to buy at http://www.fsd.it/fonts/pragma.htm
%% and is worth every penny. Any good monospace font will work fine, though.
%% Consolas or inconsolata are good alternatives.
\setromanfont{Liberation Serif} 
\setsansfont[Mapping=tex-text,Colour=AA0000]{Liberation Sans}
\setmonofont{Liberation Mono} 


%%%------------------------------------------------------------------------
%%% Local commands
%%%------------------------------------------------------------------------

%% Marginal header
%% Note: as the document goes on you may need to introduce a (gradually increasing)
%% \vspace element to keep the marginal header pleasingly aligned with the first 
%% item in the body text. Like this: \marginhead{{\vskip 0.4em}Grants}, or 
%% \marginhead{{\vskip 0.8em}Service}. Experiment as needed.
\newcommand{\marginhead}[1]{\marginpar{\textsf{{\footnotesize\vspace{-1em}\flushright #1}}}}


%% custom ampersand (font consistent with the one chosen above)
%% 
%% \newcommand{\amper}{{\fontspec[Scale=.95,Colour=AA0000]{Minion Pro Medium}\selectfont\&\,}}

%% No bullets on labels
\renewcommand{\labelitemi}{~} 

%% Custom hanging indent for vita items
\def\ind{\hangindent=1 true cm\hangafter=1 \noindent}
%\def\ind{\hangindent=18pt\hangafter=1 \noindent}
\def\labelitemi{~}
\renewcommand{\labelitemii}{~}

%%%------------------------------------------------------------------------
%%% Page layout
%%%------------------------------------------------------------------------
\pagestyle{fancy}
\renewcommand{\headrulewidth}{0pt}
\fancyhead{}
\fancyfoot{}
\rhead{{\scriptsize\thepage}}

%% git revision control footer 
\rfoot{\texttt{\scriptsize \VCRevision\ on \VCDateTEX}} % git revision info inserted via external script -- see docs for vc package for details. comment out this line if you're not using vc, and also remove the %%% This file has been generated by the vc bundle for TeX.
%%% Do not edit this file!
%%%
%%% Define Git specific macros.
\gdef\GITHash{9ff13d9b81e01641d3e05b495802089e1a2cd71f}%
\gdef\GITAbrHash{9ff13d9}%
\gdef\GITParentHashes{d0d54f9a5e5aff8fcc1df4e3d875fa3d80534a3a}%
\gdef\GITAbrParentHashes{d0d54f9}%
\gdef\GITAuthorName{Steve Koch}%
\gdef\GITAuthorEmail{stevekochscience@gmail.com}%
\gdef\GITAuthorDate{2013-02-17 21:13:01 -0700}%
\gdef\GITCommitterName{Steve Koch}%
\gdef\GITCommitterEmail{stevekochscience@gmail.com}%
\gdef\GITCommitterDate{2013-02-17 21:13:01 -0700}%
%%% Define generic version control macros.
\gdef\VCRevision{\GITAbrHash}%
\gdef\VCAuthor{\GITAuthorName}%
\gdef\VCDateRAW{2013-02-17}%
\gdef\VCDateISO{2013-02-17}%
\gdef\VCDateTEX{2013/02/17}%
\gdef\VCTime{21:13:01 -0700}%
\gdef\VCModifiedText{\textcolor{red}{with local modifications!}}%
%%% Assume clean working copy.
\gdef\VCModified{0}%
\gdef\VCRevisionMod{\VCRevision}%
 line above.

%%%------------------------------------------------------------------------
%%% Address and contact block
%%%------------------------------------------------------------------------
\begin{minipage}[t]{2.95in}
 \flushright {\footnotesize University Libraries and CARC \\ Centennial Library MSC05-3020 \\ University of New Mexico \\ \vspace{-0.05in} Albuquerque, \textsc{NM} 87131}  
  
\end{minipage}
\hfill     
%\begin{minipage}[t]{0.0in}
% dummy (needed here)
%\end{minipage}
\hfill
\begin{minipage}[t]{1.7in}
  \flushright \footnotesize Cell: \mycell \\ 
  Google: \mygoogle  \\ 
  {\scriptsize  \texttt{\href{mailto:\myemail}{\myemail}}} \\
  {\scriptsize  \texttt{\href{\myweb}{linkedin.com/in/stevekoch}}}
\end{minipage}


\medskip

%% Name 
\noindent{\Large {\textsc{Steven J. Koch}}}
\reversemarginpar

\medskip       

%% Education

\marginhead{Education}

\noindent\emph{Cornell University, Ithaca, NY \vspace{0.01in}}

\ind 2003. Ph.D, Physics (Biophysics Minor). Dissertation: \emph{Probing protein-DNA interactions by unzipping single DNA molecules with a laser trapping microscope}. %\vspace{-0.1in}  

\ind 2000. M.S., Physics (Biophysics Minor). 


\medskip
\noindent\emph{University of Michigan, Ann Arbor\vspace{0.02in}}

\ind 1996. B.S., Physics (Honors). \vspace{0.01in}

\bigskip

%% Mbio Skills

\marginhead{Molecular \mbox{Biology} Skills}

\noindent Designing and building labeled DNA constructs, single-molecule manipulation, routine protein expression and purification, routine DNA analysis (PCR, gel extraction, etc.), dynamic light scattering, enzyme activity assays. %\vspace{-0.1in}  

\bigskip

%% Other Skills

\marginhead{Other Skills}

\noindent Microscopy, hardware automation, statistical data analysis, Python, LabVIEW, parallel computing, Hadoop, image processing / tracking, Markov Chain Monte Carlo analysis, machine learning, research data management, interdisciplinary research, scientific writing, teaching (physics, biophysics, HPC, research data management). %\vspace{-0.1in}  

\bigskip

%% Appointments
\medskip
\marginhead{Appointments}

\noindent\emph{University of New Mexico, Albuquerque, NM \vspace{0.01in}}

\ind 2014 Jan - Present. Visiting Applications Scientist (CARC)

\ind 2013 - Present. Research Data Scientist (Lecturer III, Univ. Libraries)

\ind 2006–2013 May. Assistant Professor of Physics and Astronomy.      

\medskip
\noindent\emph{Sandia National Labs / CINT, Albuquerque, NM \vspace{0.01in}}

\ind 2004–2006. CINT Distinguished Postdoctoral Fellow.

\ind 2003–2004. Postdoctoral Appointee.

\medskip

\noindent\emph{Cornell University, Ithaca, NY \vspace{0.01in}}

\ind 1997–2003. Research Assistant, Physics.

\ind 1996–1997. Teaching Assistant, Physics.

\bigskip

 
%% Publications
\marginhead{{\vskip 0.3em}Publications}
\medskip
%\bigskip
\noindent\emph{Journal articles \vspace{0.05in}}
 
%% Use revnumerate environment if numbered publications are needed. 
%% (Include it above in the preamble).
%% \renewcommand{\labelenumi}{\textsc{a}\theenumi.}
%% \begin{revnumerate}

\ind Maloney A, Herskowitz LJ, Koch SJ. 2014. ``\href{https://goo.gl/L7Fema}{Effect of 2-H and 18-O water isotopes on kinesin-1 gliding assay}.'' \emph{PeerJ} 2:~e284. {\scriptsize  \texttt{\href{https://goo.gl/L7Fema}{https://goo.gl/L7Fema}}}


\ind Olendorf R, Koch S. 2012. ``\href{https://goo.gl/6or0Tn}{Beyond the low hanging fruit: Data services and archiving at the University of New Mexico}.'' \emph{Journal of Digital Information} 13:1. {\scriptsize  \texttt{\href{https://goo.gl/6or0Tn}{https://goo.gl/6or0Tn}}}


\ind Maloney A, Herskowitz LJ, Koch SJ. 2011. ``\href{http://goo.gl/SFHEs}{Effects of surface passivation on gliding motility assays}.'' \emph{PLoS ONE} 6:~e19522. {\scriptsize  \texttt{\href{http://goo.gl/SFHEs}{http://goo.gl/SFHEs}}}


\ind Liu H, Spoerke ED, Bachand M, Koch SJ, Bunker BC, Bachand GD. 2008.\newline ``\href{http://goo.gl/5ZM65}{Biomolecular Motor-Powered Self-Assembly of Dissipative Nanocomposite Rings}.'' \emph{Advanced Materials} 20:~4476–4481. {\scriptsize  \texttt{\href{http://goo.gl/5ZM65}{http://goo.gl/5ZM65}}}


\ind Xia D, Gamble TC, Mendoza EA, Koch SJ, He X, Lopez GP, Brueck SRJ . 2008. ``\href{http://goo.gl/F0RiL}{DNA Transport in Hierarchically-Structured Colloidal-Nanoparticle \newline Porous-Wall Nanochannels}.''
 \emph{Nano Letters} 8:~1610–1618. \newline {\scriptsize  \texttt{\href{http://goo.gl/F0RiL}{http://goo.gl/F0RiL}}}


\ind Rivera SB, Koch SJ, Bauer JM, Edwards JM, Bachand GD. 2007. ``\href{http://goo.gl/LwTS3}{Temperature dependent properties of a kinesin-3 motor protein from Thermomyces lanuginosus}.'' \emph{Fungal Genetics and Biology} 44:~1170–1179. {\scriptsize  \texttt{\href{http://goo.gl/LwTS3}{http://goo.gl/LwTS3}}}


\ind Koch SJ, Thayer GE, Corwin AD, de Boer MP. 2006. ``\href{http://goo.gl/8mSvW}{Micromachined piconewton force sensor for biophysics investigations}'' \newline \emph{Applied Physics Letters} 89:~173901. {\scriptsize  \texttt{\href{http://goo.gl/8mSvW}{http://goo.gl/8mSvW}}} \newline Featured in: \href{http://goo.gl/2m7LF}{“Physics News Update” December 2006} and \href{http://goo.gl/lYdCy}{Physics Today-Physics Update, February 2007}  


\ind Koch SJ, Wang MD. 2003. ``\href{http://goo.gl/XanG0}{Dynamic force spectroscopy of protein-DNA interactions by unzipping DNA}.'' \emph{Physical Review Letters} 91:~028103. \newline{\scriptsize  \texttt{\href{http://goo.gl/XanG0}{http://goo.gl/XanG0}}} Featured in: “Using powers of physics to unlock biology's secrets,” Dallas Morning News, August 04, 2003, p. E1 and \href{http://goo.gl/2m7LF}{Daily Times}. Also in Spectroscopy Now, Analytical Separations News, Bionity News, and others.


\ind Koch SJ, Shundrovsky A, Jantzen BC, Wang MD. 2002.  ``\href{http://goo.gl/G1AVA}{Probing protein-DNA interactions by unzipping a single DNA double helix}.'' \emph{Biophysical Journal} 83:~1098–1105. {\scriptsize  \texttt{\href{http://goo.gl/G1AVA}{http://goo.gl/G1AVA}}}


\ind Lawes G, Zassenhaus GM, Koch SJ, Smith EN, Reppy JD, Parpia JM. 1998. ``\href{http://goo.gl/IOI0J}{Reduction in vibrational noise from continuously filled 1 K pots}.''  \emph{Review of Scientific Instruments}
69:~4176–4178. {\scriptsize  \texttt{\href{http://goo.gl/IOI0J}{http://goo.gl/IOI0J}}}



%\end{revnumerate}
%\newpage
\bigskip

\noindent\emph{Book chapters \vspace{0.05in}}
% \renewcommand{\labelenumi}{\textsc{c}\theenumi.}
% \begin{revnumerate}

\ind Bradley J-C, Lang AS, Koch SJ, Neylon C. 2011. ``Collaboration Using Open Notebook Science in Academia'' in S. Ekins, M. A. Z. Hupcey, and A. J. Williams (Eds.) \emph{Collaborative Computational Technologies for Biomedical Research}. Wiley. ISBN: 0470638036 {\scriptsize  \texttt{\href{http://goo.gl/SiBLW}{http://goo.gl/SiBLW}}}

%\end{revnumerate}

\bigskip 
 
%\newpage
\noindent\emph{Patent application \vspace{0.05in}}

%\renewcommand{\labelenumi}{\textsc{r}\theenumi.}
%\begin{revnumerate}
\ind Koch SJ and Herskowitz LJ. 2009. "Shotgun DNA mapping by unzipping." \emph{Application pending}. Filed March 2, 2010. USPTO  Application No. 12/715,936 {\scriptsize  \texttt{\href{http://goo.gl/zKPH3}{http://goo.gl/zKPH3}}}

% %\end{revnumerate}

\bigskip 
\noindent\emph{Other writings \vspace{0.05in}}

% \ind \textbf{Blogs}


\ind Guest Blogs, Interviews: "\href{http://goo.gl/5jvJY}{Opening Science}," NextBio's Blog, June 4, 2009. "\href{http://goo.gl/qvAuU}{ScienceOnline2010 - Interview with Steve Koch}," \emph{A Blog Around the Clock} by Bora Zivkovic, January 3, 2011.

\ind Contributions (interviews) to news articles: "\href{http://goo.gl/N8INQ}{Open research casts doubt on arsenic life}," Erika Check Hayden, Nature News, August 9, 2011. "\href{http://goo.gl/bQOom}{Digital Upgrade: How to choose your lab’s next electronic notebook},” Amber Dance, The Scientist, May 1, 2010. "\href{http://goo.gl/QEtg7}{Scientists Embrace Openness},” Chelsea Wald, Science Careers, April 9, 2010. "\href{http://goo.gl/Dx9WU}{Science 2.0: You Say You Want a Revolution},” Randy Barrett, HHMI Bulletin, November 2008.   

\ind Open lecture materials: \href{http://goo.gl/2L7kC}{Physics 102 (Conceptual Physics)}: 24 sets of PowerPoint slides (75 minutes lectures).  Total slides: 626. Total lecture views: 31,375. Total lecture downloads: 764. Most downloaded lecture (136 downloads): “\href{http://goo.gl/FCsfM}{Waves, Sound, Interference, Resonance}.” 

\ind Open Data: We created two classes of large data sets, (a) microtubule gliding motility assays and (b) single-molecule DNA stretching and unzipping.  Almost all of these data sets were publicly available on our lab server (decommissioned) and were CC0 / Public Domain licensed. 

\ind Open protocols on OpenWetWare: Authored by Steven J. Koch and students in research lab. Creative Commons (CC) Attribution (BY) Share Alike (SA) license. Six major protocols posted, along with thousands of open notebook science entries.  For major protocols, 22,209 total page views.  Most popular (8,287 page views) "\href{http://goo.gl/nGmJb}{Kinesin Gliding Motility Assay}” by Dr. Andy Maloney (primary author) and Steven J. Koch.  


%% Presentations
\marginhead{{\vskip 0.4em}Invited Talks} %\newline (Since 2004)}
\medskip

\ind 2011 October 25. "Open (reproducible) data," with copanelists V. Stodden and E. Kansa. Open Access Week, Open Data Session.  University of Arizona, Tucson, AZ.

\ind 2011 October 17. "Shifting Roles in a Changing Culture: The Future of Academic Scholarship," with three other panelists at the Scholarly Communications Symposium, University of New Mexico, Albuquerque, NM.

\ind 2011 October 6. "Effects of water isotopes on kinesin gliding speeds and motor stability.  Deuterium effects on life.  Open Science progress." \href{http://goo.gl/XRjWQ}{Center for Integrated Nanotechnology (CINT) Colloquium}, Los Alamos, NM.

\ind 2011 April 22. "Open Data Successes and Challenges," University of New Mexico Cyberinfrastructure Day, Albuquerque, NM. 

\ind 2011 January 14. "Data Discoverability: Institutional Support Strategies," copanelist with K. Deards and M. Keener.  ScienceOnline2011, Research Triangle, NC.

\ind 2010 October 18. "Open Notebook Science in Undergraduate Teaching." Open Access Week Colloquium.  University of New Mexico, Albuquerque, NM.

\ind 2010 May 20. "Osmotic stress and water isotope effects in kinesin-1 gliding motility assays."  Thermodynamics and Kinetics of Molecular Motors Workshop, Santa Fe, NM.

\ind 2010 January 15. "Open Notebook Science," co-panelist with J.-C. Bradley and C. Neylon.  ScienceOnline2010, Research Triangle Park, NC.

\ind 2009 September 11. "Biophysical studies of the molecular motor kinesin." Chemistry Department Colloquium, University of New Mexico, Albuquerque, NM.

\ind 2008 April 12. "Unzipping single DNA molecules to probe protein-DNA interactions." Society of Physics Students Southwestern Zone Meeting, Albuquerque, NM.

\ind 2005 August 29. "Probing site-specific protein-DNA interactions by mechanically unzipping single DNA molecules." Nanoelectronics and Dynamics of DNA International Workshop, sponsored by Osaka University and LANL, Honolulu, HI.

\ind 2002 October 16. "Probing protein-DNA interactions by unzipping single DNA molecules." Graduate Student Symposium, Cleveland Clinic Foundation / Lerner Research Institute, Cleveland, OH.
%\end{revnumerate}

\bigskip

%\newpage

\marginhead{{\vskip 0.4em}Contributed \newline Conference \newline Presentations}
\medskip
\noindent\emph{Presenting author in \textbf{bold} \vspace{0.05in}}

\ind 2010 May 19-22. “Stochastic simulation for modeling kinesin-1.” \textbf{Herskowitz LJ}, Salvagno AL, Maloney A, Josey BP, Koch SJ. Invited poster at Conference on Thermodynamics and Kinetics of Molecular Motors, Santa Fe, NM.

\ind 2010 May 19-22. "Surface passivation and speed effects of molecular motor protein assays." \textbf{Maloney A}, Herskowitz LJ, Koch SJ. Invited poster at Conference on Thermodynamics and Kinetics of Molecular Motors, Santa Fe, NM.

\ind 2010 February. "Open-source stochastic simulation for modeling kinesin-1 kinetics." \textbf{Herskowitz LJ}, Maloney A, Black BD, Josey BP, Salvagno AL, Koch SJ. Biophysical Society Annual Meeting, San Francisco, CA.

\ind 2010 February. "Application of shotgun DNA mapping to yeast genomic DNA shotgun clones." \textbf{Salvagno AL}, Herskowitz LJ, Maloney A, Trujillo K, Le LN, Koch SJ. Biophysical Society Annual Meeting, San Francisco, CA.

\ind 2010 February. "Surface Passivation for Molecular Motor Protein Assays." \textbf{Maloney A}, Black BD, Herskowitz LJ, Salvagno AL, Le LN, Josey BP, Koch SJ. Biophysical Society Annual Meeting, San Francisco, CA.

\ind 2009 November. "Surface passivation for molecular motor protein assays." \textbf{Maloney A}, Black BD, Salvagno AL, Herskowitz LJ, Josey BP, Koch SJ. 2009 Chemical and Biological Defense Science and Technology Conference, Dallas, TX.

\ind 2009 November. "Effect of osmotic stress and D2O on kinesin activity." Maloney A, Black BD, Salvagno AL, Herskowitz LJ, Josey BP, \textbf{Koch SJ}. 2009 Chemical and Biological Defense Science and Technology Conference, Dallas, TX.

\ind 2009 February. "Proof of principle for shotgun DNA mapping by unzipping." \textbf{Herskowitz LJ}, Salvagno AL, Le LN, Koch SJ.  Biophysical Society Annual Meeting, Boston, MA.

\ind 2009 February. "Mapping Nucleosome-DNA Interactions on Single Molecules of Chromatin Isolated from Living Cells." \textbf{Ramallo Pardo DF}, Trujillo KM, Hillyer C, Osley MA, Koch SJ.  Biophysical Society Annual Meeting, Long Beach, CA.

\ind 2008 March. "Dynamic self-assembly of nanocomposite ring structures through the interaction of thermodynamic and energy-dissipating processes." \textbf{Liu H}, Spoerke E, Bachand M, Koch SJ, Bunker B, Bachand GD.  American Physical Society March Meeting, New Orleans, LA.

\ind 2007 March. "Calibration of Micromachined Force Sensors by Gravitational Force on Precision Microspheres." \textbf{Koch SJ}, Thayer GE, Corwin AD, de Boer MP. American Physical Society March Meeting, Denver CO.

\ind 2006 March. "Micromachined piconewton force sensor for biophysics investigations." \textbf{Koch SJ}, Thayer GE, Corwin AD, de Boer MP. American Physical Society March Meeting, Baltimore, MD.

\ind 2006 February. "Biomolecular Transport Systems: Building a Foundation for Adaptive Nanomaterial and Device Assembly." \textbf{Trent A}, Koch SJ, Thayer GE, Bachand GD. Biophysical Society Annual Meeting, Salt Lake City, UT.

\ind 2005 February. "Site Specific Protein-DNA Binding Kinetics Probed by Unzipping DNA with an Electromagnetic Force Instrument." \textbf{Koch SJ}, Thayer GE, Martin JE, Bunker BC, Werner JH, Goodwin PM, Keller RA, Bachand GD. Biophysical Society Annual Meeting, Long Beach, CA.

\ind 2004 March. "Active and Dynamic Nanomaterials Based on Active Biomolecules." \textbf{Koch SJ}, Rivera SB, Boal AK, Edwards JM, Bauer JM, Manginell RP, Liu J, Bunker BC, Bachand GD. American Physical Society March Meeting, Montreal, Quebec, Canada.

\ind 2004 February. "Kinesin from Thermomyces lanuginosus Displays Novel Kinetic Properties and Fast Velocities at High Temperature." \textbf{Rivera SB}, Bauer JM, Koch SJ, Edwards JM, Bachand GD. Biophysical Society Annual Meeting, Baltimore, MD.

\ind 2003 February. "Unzipping Force Analysis of Protein Association (UFAPA)." \newline \textbf{Wang MD}, Koch SJ, Shundrovsky A, Jantzen BC. Biophysical Society Annual Meeting, San Antonio, TX.

\bigskip

\marginhead{{\vskip 0.4em}Grants and \newline Awards}
\medskip
  
\ind 2009-2012. Defense Threat Reduction Agency (\textsc{DTRA}) (\$1,452,000). Susan A. Atlas (Lead PI), Steven J. Koch (co-PI), Steven Valone (co-PI). HDTRAI-09-1-0018 Coupled Atomistic Modeling and Experimental Studies of Energy Transduction and Catalysis in the Molecular Motor Protein Kinesin.

\ind 2010. Addgene (\$5,000). Steven J. Koch (PI), Andy Maloney (co-awardee). Addgene Resource Sharing Award in recognition of graduate student A. Maloney's open science contributions.

\ind 2007-2010. American Cancer Society (\textsc{ACS}) (\$42,500).  Steven J. Koch (PI), Janet Oliver (PI of ACS IRG grant \#IRG-92-024). Single-Molecule Analysis of DSB Repair Events in Vivo.

\ind 2011-present. National Science Foundation (\textsc{NSF}). Marek Osinski (PI). Steven Koch mentor on NSF 1063142 REU Site: Nanophotonics at the University of New Mexico.

\ind 2010-present. National Institutes of Health (\textsc{NIH}). Janet Oliver and Abhaya Datye (co-PIs). Steven Koch project team member on NIH 5R25CA153825 Integrative Cancer Nanoscience and Microsystems (IC-NSMS) Training Center.

\ind 2004-2006. CINT Postdoctoral Fellowship.  Sandia National Labs.

\ind 2000-2003. Molecular Biophysics Training Grant. NIH / Cornell University.

\ind 1996-2000. GAANN TA/RA Award.  US Department of Education / Cornell University.

\ind 1996. Honorable Mention NSF Graduate Research Fellowship.

\ind 1996. Sigma Pi Sigma, Physics Honor Society. University of Michigan.

\ind 1995. Phi Beta Kappa. University of Michigan.

\ind 1995. James B. Angell Scholar (two consecutive 4.0 semesters). University of Michigan.

\ind 1993-1996. Sharon Naughton-Briggs Memorial Scholarship.  University of Michigan.

\bigskip 

\newpage

%% Teaching
\medskip
\marginhead{Teaching}

\noindent\emph{Doctoral Advisement as Committee Chair \vspace{0.01in}}

\ind Roger (Andy) Maloney, Ph.D. in Physics, May 2011. "Experimental protocols for and studies of the effects of surface passivation and water isotopes on the gliding speed of microtubules propelled by kinesin-1."

\ind Lawrence J. Herskowitz, Ph.D. in Physics, December 2011. "Kinetic and Statistical Mechanical Modeling of DNA Unzipping and Kinesin Mechanochemistry."

\ind Pranav Rathi, Ph.D. in Optical Sciences and Engineering, May 2013.  \emph{Optical tweezers design, construction, and calibration. Study of DNA mechanics in regular and heavy water.}

\ind Anthony L. Salvagno, Ph.D. in Physics, May 2013. \emph{Shotgun DNA mapping by unzipping. Effects of deuterium on biomolecular interactions and in living cells.}       

\medskip
\noindent\emph{Doctoral Advisement as Committee Member \vspace{0.01in}}

\ind Michael Pochet, Ph.D. in Electrical and Computer Engineering, July 2010. "Characterization of the Dynamics of Optically-Injected Nanostructure Lasers."

\ind Douglas Read, Ph.D. in Chemical and Nuclear Engineering, July 2010. "Field-structured chemiresistors: tunable sensors for chemical-switch arrays."

\medskip

\noindent\emph{Masters Advisement as Committee Member \vspace{0.01in}}

\ind Godwin Amo-Kwao, M.S. in Physics, December 2011.

\ind Pamela Langner-Seamster, M.S. in Biomedical Sciences, December 2011.

\ind Darcy Kruse, M.S. in Physics, May 2011.

\ind Thomas Gamble, M.S. in Chemical and Nuclear Engineering, December 2009.

\medskip

\noindent\emph{Undergraduate Research Advisement \vspace{0.01in}}

\ind Alexandra Haddad (UNM ECE), Kenji Doering (U. Washington Physics), Brian Josey (UNM Physics), Linh Le (UNM Physics), Patrick Jurney (Portland U. MechE), Diego Ramallo Pardo (UNM PREP Fellow), Caleb Morse (UNM ECE), Athanasios Manole (UNM Biochemistry)

\medskip

\noindent\emph{Classroom Teaching \vspace{0.01in}}

\ind Physics 102 (Conceptual Physics): five semesters, 604 students. Physics 307L (Junior Modern Physics Lab): five semesters, 75 students. Physics 500 (Biophysics Seminar): two semesters, 18 students. Physics 581 / BME 544 (Mechanics and Thermodynamics of Molecular Components in Cells): one semester, 7 students.

\bigskip

\marginhead{{\vskip 0.9em}Service to the \newline Profession}
\medskip

\medskip

\ind Academic Editor (2010–2012), \emph{PLoS ONE}.

\ind Academic Editor (2010–2012), \emph{BiomedCentral Research Notes.}

\ind Referee, \emph{PLoS ONE}, \emph{BiomedCentral Research Notes},
\emph{European Biophysical Journal}, \emph{American Society of Mechanical Engineers}, \emph{Physical Review Letters}, \emph{Nucleic Acids Research}, Foundation for Fundamental Research on Matter (Netherlands), Department of Energy SBIR program.

\ind 2011. Consulting for astrobiology episode of NOVA.

\ind 2010-2011. Jude for Open Notebook Science Challenge.

\ind 2009-2012. Member of Membership Committee, Biophysical Society.

\ind 2007-2010. Member, Steering Committee and PI Leadership Team, OpenWetWare. 

\ind 2006-2012. Member of numerous department and university committees, Department of Physics and Astronomy, University of New Mexico.

\ind 2006-2013. Community Outreach: Kindergarten outreach ("Light, Vision, Color"), Central New Mexico Science and Engineering Research Challenge Judge, Cleveland Middle School Science Fair Judge, Menaul Middle School Inreach, Adult Science Inreach ("Winter Lab Fest"), NNIN Summer Nano Camp Guest Speaker, New Mexico MESA North Central Fall Design Judge.

\end{document}
